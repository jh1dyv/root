% 6.3.3.order.tex
%	Last update: 2018/06/21 F.Kanehori
%\newpage
\subsubsection{order.pl}
\label{subsubsec:order}

\medskip
\noindent
Visual Studio (devenv) が出力したログを、モジュール単位かつスレッド順に並べ替える。
実際の処理はライブラリ(\tt{dailybuild\_lib.pm})にある
\tt{read_log()}が行なう(\REF{subsubsec}{dailybuildlib})。

\bigskip
\noindent
起動方法
\Cmnd{4pt}{perl order.pl [options] [infile]}{-8pt}

\begin{CmndOpts}[6em]
    \CmndOpt{-o outfile}{出力ファイル名 (stdout\NOTE)}
    \CmndOpt{-p patt_id}{モジュール識別パターン番号}
    \CmndOpt{-D n}{デバッグレベル設定}
\end{CmndOpts}

\begin{CmndArgs}[6em]
    \CmndArg{infile}{入力ファイル名 (stdin\NOTE)}
\end{CmndArgs}

\begin{Process}
\begin{enumerate}
  \item	\tt{dailybuild\_lib.pm}のライブラリ関数\tt{read\_log()}を用いて
	入力ファイルの読み込み処理を行なう。
	返されるのは(\tt{\%modules, @modules, @lines})の3つ組である。\\
	\vspace{-.7\baselineskip}
	\begin{narrow}[10pt]\small{\begin{tabular}{ll}
	    \tt{\%modules}
		& モジュール名をキーとした連想配列(要素は\tt{\%threads})\\
	    \tt{\%threads}
		& スレッド番号をキーとした連想配列(要素は入力データ行の配列)\\
	    \tt{@modules}
		& モジュール名の配列(出現順)
	\end{tabular}}\end{narrow}
	\medskip

  \item	入力データ行を\tt{\%threads}からモジュール順、スレッド順に取り出して、
	\tt{outfile}(指定がなければ標準出力)に書き出す。

\end{enumerate}
\end{Process}

% end: 6.3.3.order.tex
