Base\index{Base}\KLUDGE モジュールは基本機能の集合体です.
\KLUDGE 以下では項目別にそれらについて解説します.

\section{\KLUDGE 行列・ベクトル演算}

\begin{table}[t]
\caption{Matrix and vector classes}
\label{table_matrix}
\begin{center}
\begin{tabular}{ll}
\toprule
\KLUDGE クラス名							& \KLUDGE 機能					\\ \midrule
\texttt{Vec[2|3|4][f|d]}			& \KLUDGE ベクトル				\\
\texttt{Matrix[2|3|4][f|d]}			& \KLUDGE 行列					\\
\texttt{Quaternion[f|d]}			& \KLUDGE 単位クォータニオン	\\
\texttt{Affine[f|d]}				& \KLUDGE アフィン変換			\\
\texttt{Pose[f|d]}					& 3\KLUDGE 次元ベクトルとクォータニオンの複合型 \\
\bottomrule
\end{tabular}
\end{center}
\end{table}

Table\,\ref{table_matrix}\KLUDGE によく使われる行列・ベクトルクラスを示します.
\KLUDGE 末尾の整数はベクトルや行列のサイズを表し,\texttt{f}\KLUDGE ,\texttt{d}\KLUDGE はそれぞれ\texttt{float}\KLUDGE 型,\texttt{double}\KLUDGE 型に対応します.

\subsection*{\KLUDGE ベクトル}

\index{Vec2f}\index{Vec3f}\index{Vec4f}
\index{Vec2d}\index{Vec3d}\index{Vec4d}
\index{\KLUDGE べくとる@\KLUDGE ベクトル}

\KLUDGE ベクトル型は物体の位置や速度,力などの物理量を表現するために頻繁に使われます.
\KLUDGE 例えば\texttt{double}\KLUDGE 型の要素からなる3\KLUDGE 次元ベクトル\texttt{x}\KLUDGE を定義するには
\begin{sourcecode}
Vec3d x;
\end{sourcecode}
\KLUDGE とします.要素アクセスは\texttt{[]}\KLUDGE 演算子を用います.
\begin{sourcecode}
x[0];    // 0-th element
\end{sourcecode}
\KLUDGE この他,\texttt{Vec[2|3][f|d]}\KLUDGE については\texttt{.x}, \texttt{.y}, \texttt{.z}\KLUDGE でも要素アクセスできます.

\KLUDGE 任意の固定サイズのベクトルも使えます.
\texttt{float}\KLUDGE 型の10\KLUDGE 次元ベクトルは
\begin{sourcecode}
TVector<10, float> x;    // 10-dimensional float vector
\end{sourcecode}
\KLUDGE とします.
\KLUDGE 可変長ベクトルは
\begin{sourcecode}
VVector<float> x;
x.resize(10);            // can be resized at any time
\end{sourcecode}
\KLUDGE で使えます.
\textbf{\KLUDGE ただし\texttt{VVector::resize}\KLUDGE によりサイズ変更を行うと既存の内容は破棄されますので注意して下さい.}

\KLUDGE 基本的な演算は一通りサポートされています.
\begin{sourcecode}
Vec3d a, b, c;
double k;

c = a + b;               // addition
a += b;

c = a - b;               // subtraction
a -= b;

b = k * a;               // multiply vector by scalar
a *= k;

k = x * y;               // scalar product

x % y;                   // vector product (3D vector only)
\end{sourcecode}

\KLUDGE すべてのベクトル型について以下のメンバ関数が使えます.
\begin{sourcecode}
a.size();                // number of elements
a.norm();                // norm
a.square();              // square of norm
a.unitize();             // normalize
b = a.unit();            // normalized vector
\end{sourcecode}


\subsection*{\KLUDGE 行列}

\index{Matrix2f}\index{Matrix2d}
\index{Matrix3f}\index{Matrix3d}
\index{\KLUDGE ぎょうれつ@\KLUDGE 行列}

\KLUDGE 行列は平行移動や回転などの変換や,剛体の慣性モーメントを表現するために使われます.
\KLUDGE 例えば,\texttt{double}\KLUDGE 型の要素からなる$3 \times 3$\KLUDGE 行列$A$\KLUDGE は次のように定義します.
\begin{sourcecode}
Matrix3d A;
\end{sourcecode}

\KLUDGE 要素アクセスは\texttt{[]}\KLUDGE 演算子を用います.
\begin{sourcecode}
x[0][1];    // element at 0-th row, 1-th column
\end{sourcecode}

\KLUDGE 任意の固定サイズの行列も使えます.
\KLUDGE メモリ上に列方向に要素が整列した行列は
\begin{sourcecode}
TMatrixCol<2, 3, float> M;    // column-oriented 2x3 matrix
\end{sourcecode}
\KLUDGE 要素が行方向に整列した行列は
\begin{sourcecode}
TMatrixRow<2, 3, float> M;    // row-oriented 2x3 matrix
\end{sourcecode}
\KLUDGE となります.
\KLUDGE ちなみにさきほどの\texttt{Matrix3d}\KLUDGE は\texttt{TMatrixCol<3,3,double>}\KLUDGE と等価です.

\KLUDGE 可変サイズ行列は
\begin{sourcecode}
VMatrixCol<float> M;
M.resize(10, 13);             // column-oriented variable matrix
\end{sourcecode}
\KLUDGE で使えます.
\texttt{VMatrixCol}\KLUDGE では要素はメモリ上で列方向に並びます.
\KLUDGE 一方\texttt{VMatrixRow}\KLUDGE では行方向に要素が並びます.

\KLUDGE 行列型についても,ベクトル型と同様の四則演算がサポートされています.
\KLUDGE 行列とベクトル間の演算は次のようになります.
\begin{sourcecode}
Matrix3d M;
Vec3d a, b;

b = M * a;               // multiplication
\end{sourcecode}

\KLUDGE すべての行列型について以下のメンバ関数で行数および列数が取得できます.
\begin{sourcecode}
M.height();              // number of rows
M.width();               // number of columns
\end{sourcecode}

2x2, 3x3\KLUDGE 行列については以下の静的メンバ関数が用意されています.
\begin{sourcecode}
Matrix2d N;
Matrix3d M;
double theta;
Vec3d axis;

// methods common to Matrix2[f|d] and Matrix3[f|d]
M = Matrix3d::Zero();        // zero matrix; same as M.clear()
M = Matrix3d::Unit();        // identity matrix
M = Matrix3d::Diag(x,y,z);   // diagonal matrix

N = Matrix2d::Rot(theta);    // rotation in 2D

M = Matrix3d::Rot(theta, 'x');    // rotation w.r.t. x-axis
                                  // one can specify 'y' and 'z' too
M = Matrix3d::Rot(theta, axis);   // rotation along arbitrary vector
\end{sourcecode}

\subsection*{\KLUDGE アフィン変換}

\index{Affinef}\index{Affined}
\index{\KLUDGE あふぃんへんかん@\KLUDGE アフィン変換}

\KLUDGE アフィン変換は主にグラフィクスにおける変換を指定するために使用します.
\KLUDGE アフィン変換型\texttt{Affine[f|d]}\KLUDGE は4x4\KLUDGE 行列としての機能を備えています.
\KLUDGE 加えて以下のメンバ関数が使えます.
\begin{sourcecode}
Affinef A;
Matrix3f R;
Vec3f p;

R = A.Rot();           // rotation part
p = A.Trn();           // translation part
\end{sourcecode}

\KLUDGE また,よく使用するアフィン変換を生成する静的メンバが用意されています.
\begin{sourcecode}
A = Affinef::Unit();            // identity transformation
A = Affinef::Trn(x, y, z);      // translation
A = Affinef::Rot(theta, 'x');   // rotation w.r.t. x-axis
                                // one can specify 'y' and 'z' too
A = Affinef::Rot(theta, axis);  // rotation w.r.t. arbitrary axis
A = Affinef::Scale(x, y, z);    // scaling
\end{sourcecode}

\subsection*{\KLUDGE クォータニオン}

\index{Quaterionf}\index{Quaterniond}
\index{\KLUDGE くぉーたにおん@\KLUDGE クォータニオン}

\KLUDGE クォータニオンは主に物理計算における剛体の向きや回転を表現するために使います.
\KLUDGE クォータニオンは4\KLUDGE 次元ベクトルの基本機能を備えています.

\KLUDGE 要素アクセスは\texttt{[]}\KLUDGE 演算子に加えて以下の方法が使えます.
\begin{sourcecode}
Quaterniond q;
q.w;                   // same as q[0]
q.x;                   // same as q[1]
q.y;                   // same as q[2]
q.z;                   // same as q[3]
q.V();                 // vector composed of x,y,z elements
\end{sourcecode}

\KLUDGE 演算は以下のように行います.
\KLUDGE まず,クォータニオン同士の積は回転の合成を表します.
\begin{sourcecode}
Quaterniond q, q0, q1;
q0 = Quaterniond::Rot(Rad(30.0), 'x');   // 30deg rotation along x-axis
q1 = Quaterniond::Rot(Rad(-90.0), 'y');  // -90deg rotationt along y-axis

q = q1 * q0;
\end{sourcecode}
\KLUDGE つぎに,クォータニオンと3\KLUDGE 次元ベクトルとの積は,ベクトルの回転を表します.
\begin{sourcecode}
Vec3d a(1, 0, 0);
Vec3d b = q0 * a;
\end{sourcecode}
\KLUDGE このように,クォータニオンは基本的に回転行列を同じような感覚で使えます.
\texttt{Quaterniond[f|d]}\KLUDGE には以下のメンバ関数があります.
\KLUDGE まず回転軸と回転角度を取得するには
\begin{sourcecode}
Vec3d axis = q.Axis();        // rotation axis
double angle = q.Theta();     // rotation angle
\end{sourcecode}
\KLUDGE とします.
\KLUDGE また,逆回転を表す共役クォータニオンを得るには
\begin{sourcecode}
q.Conjugate();         // conjugate (reverse rotation)

Quaterniond y;
y = q.Conjugated();    // return conjugated quaternion
y = q.Inv();           // return inverse (normalized conjugate)
\end{sourcecode}
\KLUDGE とします.
\texttt{Conjugate}\KLUDGE はそのクォータニオン自体を共役クォータニオンに変換するのに対し,
\texttt{Conjugated}\KLUDGE は単位共役クォータニオンを返します.
\texttt{Inv}\KLUDGE は\texttt{Conjugated}\KLUDGE とほぼ等価ですが,戻り値のノルムが$1$\KLUDGE となるように正規化を行います.
\KLUDGE 回転を表すクォータニオンは理論上は必ずノルムが$1$\KLUDGE なので正規化は不要ですが,
\KLUDGE 実際は数値計算における誤差で次第にノルムがずれてくることがあります.
\KLUDGE このような誤差を補正するために適宜正規化を行う必要があります.

\KLUDGE 回転行列と相互変換するには以下のようにします.
\begin{sourcecode}
Matrix3d R = Matrix3d::Rot(Rad(60.0), 'z');
q.FromMatrix(R);       // conversion from rotation matrix
q.ToMatrix(R);         // conversion to rotation matrix
\end{sourcecode}
\texttt{FromMatrix}\KLUDGE は渡された回転行列\texttt{R}\KLUDGE と等価なクォータニオンとして\texttt{q}\KLUDGE を設定します.
\KLUDGE 一方\texttt{ToMatrix}\KLUDGE は,参照渡しされた\texttt{R}\KLUDGE を\texttt{q}\KLUDGE と等価な回転行列として設定します.

\KLUDGE 同様に,以下はオイラー角との相互変換を行います.
\begin{sourcecode}
Vec3d angle;
q.ToEuler(angle);      // to Euler angle
q.FromEuler(angle);    // from Euler angle
\end{sourcecode}

\KLUDGE 最後に,以下の関数は2\KLUDGE つのベクトルに対し,片方をもう片方に一致されるような回転を表すクォータニオンを求めます.
\KLUDGE 一般に2\KLUDGE つのベクトルを一致させる回転は一意ではありませんが,\texttt{RotationArc}\KLUDGE は両方のベクトルに直交する軸に関する回転,
\KLUDGE いわば最短距離の回転を求めます.
\begin{sourcecode}
Vec3d r0(1, 0, 0), r1(0, 1, 0);
q.RotationArc(r0, r1);    // rotation that maps r0 to r1 
\end{sourcecode}

\subsection*{\KLUDGE ポーズ}

\index{Posef}\index{Posed}
\index{\KLUDGE ぽーず@\KLUDGE ポーズ}

\KLUDGE ポーズは位置と向きの複合型です.
\KLUDGE 役割としてはアフィン変換に似ていますが,全部で7\KLUDGE つの成分で表現できるためアフィン変換よりもコンパクトです.
\KLUDGE ポーズは物理計算での剛体の位置と向きを表現するためなどに用います.

\texttt{Pose[f|d]}\KLUDGE 型のメンバ変数は\texttt{Pos}\KLUDGE と\texttt{Ori}\KLUDGE の2\KLUDGE つのみで,
\KLUDGE それぞれポーズの並進成分(\texttt{Vec3[f|d]})\KLUDGE と回転成分(\texttt{Quaternion[f|d]})\KLUDGE への参照を返します.
\begin{sourcecode}
Posed P;
P.Pos() = Vec3d(1, 2, 3);
P.Ori() = Quaterniond::Rot(Rad(45.0), 'x');
Vec3d p = P.Pos();
Quaterniond q = P.Ori();
\end{sourcecode}

\subsection*{\KLUDGE 初期値について}

\texttt{Vec[2|3|4][f|d]}\KLUDGE 型はゼロベクトルに初期化されます.
\texttt{Matrix[2|3][f|d]}\KLUDGE 型および\texttt{Affine[f|d]}\KLUDGE 型は単位行列に初期化されます.
\KLUDGE また,\texttt{Quaternion[f|d]}\KLUDGE は恒等写像を表すクォータニオンに初期化されます.

\section{\KLUDGE スマートポインタ}

\index{\KLUDGE すまーとぽいんた@\KLUDGE スマートポインタ}

\KLUDGE 参照カウントにもとづくスマートポインタです.
\KLUDGE 参照カウントが$0$\KLUDGE になったオブジェクトのメモリを自動的に解放するためにユーザが手動で\texttt{delete}\KLUDGE を実行する手間が省け,メモリリークの危険が低減できます.

\index{UTRefCount}
\index{UTRef}
\KLUDGE 参照カウントクラスは\texttt{UTRefCount}\KLUDGE です.
Springhead\KLUDGE のほとんどのクラスは\texttt{UTRefCount}\KLUDGE を継承しています.
\KLUDGE スマートポインタはテンプレートクラス\texttt{UTRef}\KLUDGE です.
\KLUDGE 以下に例を示します.

\begin{sourcecode}
class A : public UTRefCount{};
UTRef<A> a = new A();
// no need to delete a
\end{sourcecode}

\section{\KLUDGE その他の機能}

\subsection*{UTString}

\index{UTString}
\KLUDGE 文字列型です.現状ではstd::string\KLUDGE と等価です.

\subsection*{UTTypeDesc}

\index{UTTypeDesc}
Springhead\KLUDGE のクラスが持つ実行時型情報です.

\subsection*{UTTreeNode}

\index{UTTreeNode}
\KLUDGE ツリー構造の基本クラスです.

