% 6.1.MakeReportScriptTree.tex
%	Last update: 2018/08/20 F.Kanehori

\subsection{Script構成}
\label{subsec:MakeReportScriptTree}

\def\Anno#1{\mc{\footnotesize{ \CDots\  #1}}}
\def\AnnoRef#1#2{\Anno{#1} \footnotesize{(\ref{#2} 参照)}}

\noindent
プログラム構造
\vspace{.2\baselineskip}
\begin{narrow}
    \begin{narrow}\begin{minipage}{\textwidth}
	\dirtree{%
		.1 \Hskip{-10mm}I:/DailyBuild/Springhead/core/.
		.2 test/.
		.3 \it{MakeReportGit.bat}
			\Anno{MakeReport制御スクリプト}.
		.3 Monitoring/.
		.4 bin/.
		.5 \it{build_monitor_Git.bat}
			\AnnoRef{レポート作成メイン}{subsubsec:buildmonitorGit}.
		.5 \it{build_monitor_SVN.bat}
			\Anno{同上Subversion版 --- obsoleted}.
	}
	\medskip
    \end{minipage}\end{narrow}
\end{narrow}

\bigskip
\noindent
関連スクリプト構成

\vspace{-12pt}
\noindent
\begin{narrow}
    \begin{narrow}\begin{minipage}{.8\textwidth}
	\dirtree{%
		.1 \Hskip{-10mm}I:/DailyBuild/Springhead/core/.
		.2 test/.
		.3 Monitoring/.
		.4 bin/.
		.5 \it{filter.pl}
			\AnnoRef{diff -c形式情報作成}{subsubsec:filter}.
		.5 \it{mydiff.pl}
			\AnnoRef{diff情報作成}{subsubsec:mydiff}.
		.5 \it{order.pl}
			\AnnoRef{Visual Studioログ行並べ替え}{subsubsec:order}.
		.5 \it{base_lib.pm}
			\AnnoRef{共通ライブラリ}{subsubsec:baselib}.
		.5 \it{dailybuild_lib.pm}
			\AnnoRef{dailybuild用ライブラリ}{subsubsec:dailybuildlib}.
		.5 \it{exclude.awk}
			\AnnoRef{不要な行の削除}{subsubsec:awkscripts}.
		.5 \it{field.awk}
			\AnnoRef{特定フィールドの抽出}{subsubsec:awkscripts}.
		.5 \it{grep.awk}
			\AnnoRef{grepもどき}{subsubsec:awkscripts}.
		.5 \it{head.awk}
			\AnnoRef{headもどき}{subsubsec:awkscripts}.
	}
  \end{minipage}\end{narrow}
  \medskip
\end{narrow}

\bigskip
\noindent
外部プログラム構成

\vspace{-12pt}
\noindent
\begin{narrow}
    \begin{narrow}\begin{minipage}{.8\textwidth}
	\dirtree{%
		.1 \hspace{-10mm}I:/DailyBuild/Springhead/core/.
		.2 test/.
		.3 Monitoring/.
		.4 bin/.
		.5 \it{gawk.exe}.
		.5 \it{nkf.exe}.
	}
  \end{minipage}\end{narrow}
  \medskip
\end{narrow}

\bigskip
\noindent
関連ファイル構成

\vspace{-12pt}
\noindent
\begin{narrow}
    \begin{narrow}\begin{minipage}{.8\textwidth}
	\dirtree{%
		.1 \hspace{-10mm}I:/DailyBuild/Springhead/core/.
		.2 test/.
		.3 Monitoring/.
		.4 etc/.
		.5 \it{revision.new}
			\Anno{比較レビジョン定義ファイル}.
		.5 \it{revision.old}
			\Anno{基準レビジョン定義ファイル}.
	}
  \end{minipage}\end{narrow}
  \bigskip
  \noindent
  基準レビジョン定義ファイル
  \begin{narrow}
  	レポート作成の基準とするレビジョン(コミットID)を指定するファイル。
	次の形式のテキストファイルである。
	\begin{itemize}
	  \item	行頭から短形式のコミットIDを記述する。
	  	コミットIDの代わりに\SQuote{\tt{HEAD}}と記述したときは、
		最新のコミットIDを指定したものと解釈する。
	  \item	1つ以上の空白を置けば後ろに何を書いても無視する。
	  \item	ハッシュ文字(\SQuote{\tt{\#}})以降及び空行は無視する。
	\end{itemize}
  \end{narrow}
  \noindent
  比較レビジョン定義ファイル
  \begin{narrow}
  	レポート作成の対象となるレビジョン(コミットID)を指定するファイル。
	形式は基準レビジョン定義ファイルと同じである(上記参照)。
  \end{narrow}
\end{narrow}

% end: 6.1.MakeReportScriptTree.tex
