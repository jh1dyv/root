% 3.2.CmakeApplication.tex
%	Last update: 2019/10/07 F.Kanehori
%newpage
\subsection{7RJG1g82P0Zb5WIz}
\label{subsec:CmakeApplication}

\noindent
\cmake \KLUDGE の実行手順については、
\KQuote{\ref{subsec:CmakeLibrary} cmake\KLUDGE の実行}\KLUDGE と同じですから、
\KLUDGE そちらを参照してください。
\KLUDGE ここでは、\cmake \KLUDGE を実行することで
\KQuote{\ref{subsec:Solution} \KLUDGE 対処法}\KLUDGE で示した対策がどのように実施されるかについて述べます。

\bigskip
\noindent
\bf{\KLUDGE ビルドの最適性}
\begin{narrow}[20pt]
	\KLUDGE 組み込まれているSpringhead\KLUDGE ライブラリの各プロジェクト
	(\KLUDGE ここでは\tt{Base}\KLUDGE を例にします)\KLUDGE に対して\\
	\hspace{20pt}\SprTop{/core/src/Base/\it{\textless x64\textgreater}/%
		\it{\textless 15.0\textgreater}/Base.dir}\Note{(*1)}\\
	\KLUDGE というディレクトリを作成し、
	\AppTop{/\build /Base/Base.dir}\KLUDGE から上記ディレクトリ\Note{(*1)}\KLUDGE へ
	link\KLUDGE を張る作業を\cmake\ configure\KLUDGE 時に(\KLUDGE 自動的に)\KLUDGE 行ないます。
\end{narrow}	
\begin{narrow}[20pt]
	\thinrule{\linewidth}\\
	{\bf{\KLUDGE これに関して皆さんが行なうべきことはありませんが、
	\KLUDGE 次のことに注意してください。}}

	\medskip
	\begin{enumerate}
	  \item	\cmake \KLUDGE をした後でSpringhead\KLUDGE ソースツリー上ので上記ディレクトリ
		\Note{(*1)}\KLUDGE を削除すると、以降のビルドで\\
		\hspace{15pt}\KQuote{\tt{\small{%
		\KLUDGE エラー MSB3191 \KLUDGE ディレクトリ\Path{Base.dir/Debug/}\KLUDGE を作成できません。}}}

		\KLUDGE というエラーが発生します。

	  \item	\SprLib \KLUDGE 側で\cmake\ (configure)\KLUDGE を実行していないと
		\KLUDGE オブジェクトの共通格納領域が作成されていないため、
		\KLUDGE 前項と同じエラーが発生します。

	  \item	\KLUDGE アプリケーション側で\AppTop{/\build /Base/Base.dir}\KLUDGE を削除すると、
		\KLUDGE ビルド時にVisual Studio\KLUDGE が\build \KLUDGE 下に\Path{Base.dir}\KLUDGE を
		\KLUDGE 自動的に作成するためにビルドの最適性が崩れてしまいます
		(\KLUDGE 無駄なビルドが発生するだけで、ビルド自体は正常に行なえます)\KLUDGE 。
		\begin{narrow}[s][15pt]
		Windows\KLUDGE では実行権限の都合上link\KLUDGE をjunction\KLUDGE で実現していますが、
		explorer\KLUDGE でもcommand prompt\KLUDGE 上でも\Path{Base.dir}\KLUDGE がjunction\KLUDGE か
		\KLUDGE 通常のディレクトリかの区別がつきません。
		\KLUDGE そのためこのような事態の発見が困難になる可能性があります。
		\end{narrow}
	\end{enumerate}
	\medskip
	{\bf{\KLUDGE これらの状態を解消するためには、アプリケーション側
	(\KLUDGE もしくはSpringhead\KLUDGE ライブラリ側)\KLUDGE で再度\cmake \KLUDGE を実行する必要があります。}}

	\thinrule{\linewidth}
\end{narrow}

\bigskip
\noindent
\bf{\KLUDGE プロジェクトファイルの整合性}
\begin{narrow}[20pt]
	\KQuote{\ref{subsec:Solution} \KLUDGE 対処法}\KLUDGE でも述べたように、
	\KLUDGE プロジェクトファイル(\KLUDGE ソリューションファイルの含む。以下同様)\KLUDGE は
	Springhead\KLUDGE ライブラリのビルドツリーにあるものを最新の状態に保つことを前提として、
	\KLUDGE アプリケーション側のプロジェクトファイルは
	\KLUDGE これらSpringhead\KLUDGE 側にあるものへのlink\KLUDGE となるようにします。
	\KLUDGE このためにアプリケーションプログラムのソリューションに
	\tt{sync}\KLUDGE というターゲットを追加して、次の処理を実行させます。
	\begin{enumerate}
	  \item	\KLUDGE もしもアプリケーション側のプロジェクトファイルの内容と
		Springhead\KLUDGE 側のプロジェクトファイルの内容とが異なっていたならば、
		\KLUDGE アプリケーション側のプロジェクトファイルを
		Springhead\KLUDGE 側にコピーする。
		\begin{narrow}[s][15pt]
		\KLUDGE これは、アプリケーション側でソースファイル構成を変更
		(\KLUDGE ファイルの追加・削除)\KLUDGE を行ない、
		Visual Studio\KLUDGE 上でプロジェクトファイルを保存したとき、
		\KLUDGE またはアプリケーション側で再度\cmake \KLUDGE を実行したときです。
		\end{narrow}

	  \item	\KLUDGE アプリケーション側のプロジェクトファイルをSpringhead\KLUDGE 側の
		\KLUDGE プロジェクトファイルへのlink\KLUDGE とする。
	\end{enumerate}
	\KLUDGE ターゲット\tt{sync}\KLUDGE はアプリケーションのビルドにおいて
	\KLUDGE 必ず最初に実行されるように依存関係が設定されます。
\end{narrow}	
\begin{narrow}[20pt]
	\thinrule{\linewidth}\\
	{\bf{\KLUDGE 次のことに注意してください。}}

	\medskip
	\begin{enumerate}
	  \item	Springhead\KLUDGE 側または他のアプリケーションが実施した変更は、
		\KLUDGE アプリケーションをビルドするだけで自動的に反映されます。

	  \item	\KLUDGE 自アプリケーションで実施したソース構成の変更は、
		\KLUDGE ビルドを実施した時点でSpringhead\KLUDGE 側に反映されます。
		\KLUDGE つまり、プロジェクトファイルの整合性を保つためには、
		\KLUDGE ソース構成の変更後に少なくとも1\KLUDGE 回はビルドを実行する必要がある
		\KLUDGE ということです(\tt{sync}\KLUDGE の実行だけでもよい)\KLUDGE 。
		\begin{narrow}[s][15pt]
		\KLUDGE ソース構成を変更したらビルドするでしょうから、
		\KLUDGE このことが問題になることはほとんどないと思われます。

		\KLUDGE もしSpringhead\KLUDGE 側でプロジェクトファイルが削除されたならば、
		\KLUDGE ビルドエラー(\tt{sync}\KLUDGE でlink\KLUDGE 先のファイルが見つからない)
		\KLUDGE となります。
		
		{\bf{\KLUDGE この場合にはSpringhead\KLUDGE 側で再度\cmake \KLUDGE を実行する必要があります
		(\KLUDGE アプリケーション側では駄目)\KLUDGE 。}}
		\end{narrow}

	  \item	\tt{sync}\KLUDGE ターゲットが実行されるとプロジェクトファイルが更新される
		\KLUDGE ことがあるため、\KQuoteS \KLUDGE プロジェクトが環境外で変更された\KQuoteE \KLUDGE 旨の
		\KLUDGE メッセージが出ることがあります。
		\KLUDGE 「すべて再読み込み」としてください。
	\end{enumerate}

	\thinrule{\linewidth}
\end{narrow}	

% end: 3.2.CmakeApplication.tex
