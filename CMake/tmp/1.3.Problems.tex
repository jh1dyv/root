% 1.3.Problems.tex
%	Last update: 2019/10/03 F.Kanehori
%\newpage
\subsection{sXn8UpnPwdz1mDIw}
\label{subsec:Problems}

\def\App#1{\it{App#1\,}}
\def\App#1{\it{App#1\,}}

\noindent
\KLUDGE 前節の図\ref{fig:ApplicationTree}\KLUDGE に示したような
\KLUDGE アプリケーション\App{1}\KLUDGE の他にアプリケーション\App{2}\KLUDGE があったとして、
\KLUDGE その両方を同時に開いて作業を行なう場合を考えます。
\KLUDGE ここでは\App{1}\KLUDGE と\App{2}\KLUDGE が共通して参照する\SprLib \KLUDGE のプロジェクトについてのみ考えます。

\bigskip
\noindent
\bf{\KLUDGE ソースファイルの整合性}
\begin{narrow}[20pt]
	\KLUDGE ソースファイルの整合性には問題がありません。
	\App{1}\KLUDGE と\App{2}\KLUDGE のどちらも\SprLib \KLUDGE のソースツリーを指すようにしてありますから、
	\KLUDGE どちらで実施したソースファイルの変更も直ちに他方に反映されることになります。
	(\KLUDGE 同じファイルを参照しているのですから当然です)

	\KLUDGE  ソースファイルの追加や削除を実施したとしても
	\KLUDGE ソースファイルの整合性自体が崩れるわけではありません。
	\KLUDGE ただし、この場合には\KQuote{\bf{\KLUDGE プロジェクトファイルの整合性}}\KLUDGE の問題が発生します。
\end{narrow}

\medskip
\noindent
\bf{\KLUDGE ビルドの最適性(\KLUDGE 無駄なコンパイル)}
\begin{narrow}[20pt]
	\App{1}\KLUDGE と\App{2}\KLUDGE のビルドツリーは異なるものですから、
	\App{1}\KLUDGE でソースを変更してビルドしたとしても
	\KLUDGE そのビルド結果(\KLUDGE オブジェクトファイル)\KLUDGE がそのまま\App{2}\KLUDGE に反映されることはありません。
	\KLUDGE これらは異なるファイルです。

	\KLUDGE  \App{2}\KLUDGE でソースの変更を反映させようとすれば
	\App{2}\KLUDGE で再ビルドを行なう必要があります (\KLUDGE この時点で変更されている
	\KLUDGE ソースファイルが再コンパイルされ、オブジェクトファイルの整合性がとれます)。
	\KLUDGE 従来はオブジェクトファイルも共有されていましたから、
	\KLUDGE このような無駄なコンパイルは発生しませんでした。
\end{narrow}

\medskip
\noindent
\bf{\KLUDGE プロジェクトファイルの整合性}
\begin{narrow}[20pt]
	\KLUDGE ソースファイルの内容を修正しただけならば
	\KLUDGE プロジェクトファイルが変更されることはありません。
	\KLUDGE しかし、ソースファイルの追加や削除を行なったならば
	\KLUDGE プロジェクトファイルにも変更が及びます
	(Visual Studio\KLUDGE 上でソースの追加・削除を行なってセーブを行なった場合、
	\KLUDGE もしくは\App{1}\KLUDGE での変更後に再\cmake \KLUDGE を行なった場合)\KLUDGE 。

	\KLUDGE  プロジェクトファイルもオブジェクトファイルと同様に
	\KLUDGE ビルドツリー内に生成されますから、
	\App{1}\KLUDGE で実施した変更が\App{2}\KLUDGE に反映されることはありません。
	\KLUDGE しかも\App{2}\KLUDGE で再ビルドしたとしても\App{1}\KLUDGE での変更が反映されることはありません。
	\App{2}\KLUDGE のプロジェクトファイルは\App{1}\KLUDGE のものとは異なるファイルですから
	\App{2}\KLUDGE は依然として古い状態を残したままなのです。

	\indent
	\KLUDGE  これを解消するには\App{2}\KLUDGE で再度\cmake \KLUDGE を実行する必要がありますが、
	\KLUDGE 問題は\KQuote{\bf{\KLUDGE いつ再\cmake \KLUDGE が必要なのかが分からない}}\KLUDGE ということです。
\end{narrow}

% end: 1.3.Problems.tex
